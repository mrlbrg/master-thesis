\chapter{Background}

\section{Database System Architecture Overview}

% TODO: Mention that we don't consider writes to the log which should be the same in all systems.
% TODO: Work in equations and read/space ampl. from "A Comparison of Fractal Trees to Log-Structured Merge (LSM) Trees" https://www.cs.cmu.edu/~dga/papers/sigmod10-fractal.pdf

\begin{figure}[htpb]
  \centering
  \includegraphics[width=0.99\textwidth]{figures/db_architecture.pdf}
  \caption{The storage and access layer of a database system.}
  \label{fig:db-architecture}
\end{figure}

In the scope of this thesis we focus on a classic architecture of a single-node, disk-based database system.
The access and storage layer of a database system typically consist of a buffer manager, one or more index structures and the slotted pages that store tuples identified by \ac{TID}s, as illustrated in \autoref{fig:db-architecture}.
Since we operate in a beyond memory setting, the buffer manager is responsible for caching pages in \ac{DRAM} and loading them from external storage when needed.
Therefore, all components accessing physical data interact with the buffer manager to load and store their pages.
When a query is executed, the index is accessed by a given key (e.g. the primary key) to find the \ac{TID} of the relevant tuple.
The index is typically stored in pages, which are loaded into the buffer pool by the buffer manager.
Using the \ac{TID}, the corresponding tuple can be retrieved from the slotted pages.
The \ac{TID} encodes the page identifier and the slot number within the page.
When a tuple is updated, the corresponding page is loaded into the buffer pool, modified, and marked as dirty.
Should the buffer pool be full, the buffer manager evicts pages based on its replacement policy.
Clean, unchanged pages can be discarded, while dirty, modified pages must be written back to external storage.


\section{Index Structures}
Index structures are data structures that enable efficient access to data stored in a database.
Typically, they map a key to a constant, unique \ac{TID}. A \ac{TID} never changes for a tuple.
Keys can be arbitrary types and therefore of fixed or variable size, such as integers or strings.
We will consider both within this thesis.
When the key of a tuple changes, the index must be updated to reflect the new key.

Some key-value stores directly map keys to tuples within their index structure, omitting the indirection via \ac{TID} and slotted pages.
However, in a general purpose \ac{DBMS}, we typically want to support multiple indexes on the same data.
If we stored tuples directly in the index, we would need to update all indexes when a tuple changes.
Therefore the access and storage layer are decoupled via \ac{TID}s.

Indexes can be classified into primary and secondary indexes.
A primary index is built on the primary key of a table, which uniquely identifies each tuple.
A secondary index is built on a non-primary key, which can be non-unique.

Having sequential access to a primary key is common, for example when inserting new tuples with an auto-incrementing primary key.
However, secondary keys are often accessed randomly. For example, consider a user's email address as a secondary key.
When inserting a new user, the email address is likely to be random and not follow any specific order.
Therefore, secondary indexes often exhibit random access patterns, which can lead to inefficient access patterns in traditional index structures like a B-Tree.

\section{B-Trees}

\begin{figure}[htpb]
  \centering
  \includegraphics[width=0.99\textwidth]{figures/b_tree.pdf}
  \caption{A B+-Tree. Child pointers are repesented as arrows. Values (the \ac{TID}) are represented as bullet points •.}
  \label{fig:b-tree}
\end{figure}

B-Trees \cite{bayer1970organization} are a self-balancing tree data structure that maintains sorted data and allows for insertion, deletion, and search operations in logarithmic time, $\mathcal{O}(\log n)$.
A B-Tree is organized in fixed-size pages, called nodes. These pages are transferred transparently by the buffer manager between external storage and \ac{DRAM}.
Each node can split off a sibling once is it full. If a node is full and a new key needs to be inserted, the node splits into two nodes, and the middle key is promoted to the parent node.
Additionally, nodes can merge with a sibling if they become less than half full. For simpliticity, we omit merging of nodes in this thesis.
The tree only increases in height when the root node splits.
Each node contains between 2 and 2k entries, except for the root node, which can contain between 1 and 2k entries.
Each entry is a triple of a key, a pointer to a child node, and optionally a value (the \ac{TID}).
The entries in each node are sorted by key. On leaf nodes (nodes without children), the pointer to a child node is undefined.
An inner node (a node that is not a leaf) with k keys has k+1 children.
Each entry in an innder node separates the key space of its children.
The additional child pointer is necessary to separate the key space above the largest key in the node.
For example, consider an inner node with keys \{10, 20, 30\}.
The first child contains all keys less than 10, the second child contains all keys between 10 and 20, and the third child contains all keys between 20 and 30. 
The fourth child contains all keys greater than 30.

When searching for a key in the tree, we start at the root node.
On each node, we perform a binary search to find the appropriate pivot key and follow the corresponding child pointer.
We stop when we reach a node with the desired key.

\subsection*{B+-Trees}
When addressing B-Trees in this thesis, we actually refer to B+-Trees, a variant of B-Trees where all values are stored in the leaf nodes and internal nodes only store keys and child pointers to guide the search.
The separator keys in internal nodes may or may not occur in the data. An example B+-Tree is illustrated in \autoref{fig:b-tree}.
The lookup procedure is the same as in a B-Tree, however we always traverse the full tree from root to leaf to find a key.
Not only does this simplify the B-Tree logic, it also increases the fanout of inner nodes, leading to a lower tree height and therefore fewer \ac{IO} operations for lookups since less pages are involved in reaching the leaf level.
Also, it allows for efficient range queries by scanning the leaf nodes in order.
Due to its excellent lookup performance, support for range queries, and simplicity, B-Trees are the dominant data structure for external storage \cite{mdbs2024slides}.

% Slotted Node Layout here or in Implementation?

\subsection*{Node Size \& Fanout}
\label{sec:node-size-fanout}
The node size is a crucial parameter in the design of a B-Tree, as it affects the height of the tree.
The height of a B-Tree is $\log_f(n)$, where $f$ is the fanout (the number of children per node) of the tree and $n$ is the number of keys in the tree.
The node size determines how many entries fit into a node, which directly impacts the fanout $f$ of the tree and therefore its height.
Larger nodes lead to more entries per node, increasing the fanout. 
When we can address more children per node, we need fewer levels in the tree to address the same number of keys.
Therefore, larger nodes lead to a more shallow tree. 
Since every lookup requires a traversal from the root to a leaf node, a more shallow tree leads to fewer pages involved in the lookup.
Thus, larger nodes lead to fewer \ac{IO} operations per lookup.
Additionally, since we need fewer distrinct pages, we induce less page management overhead in the buffer manager.
This is particularly beneficial for analytical workloads, which often perform large scans and are interested in large parts of the data.
However, workloads that perform many updates and point queries, we are often only interested in a small portion of the page.
As a result, larger nodes lead to more \ac{IO} amplification, as we read and write significantly more data than necessary to perform the operation.
Research on modern \ac{SSD} shows that a good compromise is given for page sizes of 4 KB \cite{haas2023modern}.
% TODO: Make sure this is still correct by the end of the thesis:
While we will be investigating different node sizes in our evaluation, this parameter is not the primary focus of this thesis.
Instead, we focus on reducing \ac{IO} amplification in B-Trees at any page size, but larger nodes profit more significantly from our approach.

\section{External Storage Characteristics}
For some time, in-memory database systems like Hyper \cite{kemper2011hyper} have gained popularity due to the decreasing cost of \ac{DRAM}.
However, that trend has reversed recently, as \ac{DRAM} prices have stagnated \cite{haas2023modern} and \ac{SSD} price-performance-ratio have improved significantly \cite{leis2024leanstore}.
Therefore, modern database systems are designed to operate efficiently on external storage and since index structures are the performance-critical component, out-of-memory indexing has become a key consideration again.
B-Trees have been the dominant index structure for out-of-memory indexing, since their high fanout minimizes the number of \ac{IO} operations.

Historically, hard disks were the dominant storage medium.
Hard disks have a significant imbalance in latency between random and sequential \ac{IO} due to their mechanical nature.
While \ac{SSD} have a smaller difference between random and sequential \ac{IO}, they still exhibit asymmetric performance \cite{haas2023modern}.
Therefore, to amortize the cost of random \ac{IO}, database systems and their index structures are designed to access data in pages of multiple kilobytes instead of individual tuples.
While we will be referencing disk-based systems throughout this thesis, we speak of systems operating on external storage, which can be either disk-based or flash-based.

% SSD Wear Leveling? Garbage Collection? Wearout? https://www.vldb.org/pvldb/vol18/p4295-haas.pdf

\section{Write Amplification}
Write amplification is the ratio of the amount of data written to storage versus the amount of logical data written by the user.
For example, if the B-Tree updates an entry of 64 B, but needs to write a full page of 4 KB to storage, the write amplification is 4096 / 64 = 64.
Write amplification $WA$ is formally defined as:

\[
WA = \frac{Bytes Written Physically}{Bytes Written Logically}
\]

There are multiple layers of write amplification in a database system, which we need to differentiate from each other.

\inlinesection{Application Layer.}
At the application layer, we consider an external end user interacting with the database system.
When an end user inserts a tuple through a query, the database system must insert the tuple in the table itself, as well as all indexes that serve to access the tuple later.
The bytes written logically are the bytes of the new tuple.
Consequently, we already write more bytes than the user inserted.
Additionally, updating a B-Tree index might cause structural changes such as node splits or merges that create new nodes, delete new nodes and update the parent nodes.
Thus, a database system inherently comes with write amplification. Write overhead to manage data efficiently.
This is however not the focus of this thesis; we assume all updates to the B-Tree structure as necessary and focus on reducing write amplification in the index layer.

\inlinesection{Index Layer.}
At the index layer, we consider the B-Tree as the user of pages.
When a B-Tree entry is updated, the bytes written logically include not only the updated key but also any additional metadata required to maintain the tree structure.
This can include information about node splits, merges, and the promotion of keys to parent nodes.
The writes are amplified by the B-Tree's need to rewrite entire pages even if only a small portion of the page changed.
As mentioned in \autoref{sec:node-size-fanout}, the node size directly impacts the write amplification at this level.
This is the write amplification we focus on in this thesis, by minimizing the number of pages written to storage for a given set of updates (see \autoref{chap:method}).

\inlinesection{Physical Layer.}
At the physical layer, we consider the database system as the user (the host) of the physical storage device.
When the database system writes a page to storage, the bytes written logically are the size of the page.
However, due to the characteristics of the storage device, the actual bytes written physically can be larger.
\ac{SSD} typically operate in larger units called blocks, which consist of multiple pages.
When a page is updated, the entire block containing that page must be rewritten, leading to write amplification.
Additionally, when garbage collection is performed, valid pages within a block must be copied to a new block before the old block can be erased and reused.
Recent research have observed write amplification factors up to 10x on modern \ac{SSD} \cite{haas2025ssd}.
Consequently, a page write of 4 KB can lead to physical writes of up to 40 KB on the device, using up valuable bandwidth and wearing out the device faster.
While we do not focus on hardware-level write amplification in this thesis, it shows the importance of reducing write amplification at the database system level.
Any unnecessary write at the database system level is multiplied by the storage device.

% See Bepsilon paper for more analytical model