\chapter{Method}
\label{chap:method}
% Obervation: Write amplification in B-Trees is caused at the point of eviction. This is the point in time to buffer to defer a write.
% Observation: The data structure itself should remain free of buffering logic to not compromise concurrency.
% Idea: Introduce a layer between the data structure and the storage manager that buffers writes and applies them in batches.

% Learned from related work that we want to turn small writes into large sequential writes by batching them.

% Figure: Diagram of the architecture with the new layer.

\section{Design Goals}
\section{High-Level Description of the Optimization}
\section{Data Structure Modifications}
\section{Algorithms for Insertion, Deletion, and Rebalancing}
\section{Theoretical Implications on Write Amplification}
