\chapter{Discussion and Future Work}
\section{Summary of Findings}
% “Our approach reduces write amplification by 40–60% while maintaining within 5% of baseline read latency.”11,848
\section{Limitations}
% Size of the delta tree. 
\section{Future Directions}
% Other data strcutured than B-Tree that have more batching effect
% Use levelling that keeps inserting into another level if write amplification is too high. Better batching.
% When the Delta Tree gets too big, we just transfer random writes from one tree to another.
% Therefore it only makes sense to track small changes. If a page has a lot of changes, it is better to write it out and not track it anymore.
\section{Potential Applications}

% Bloom filter in memory to lookup the delta tree if necessary only when loading a B-Tree node.
% Results could look very different on a buffer manager that evicts pages smarter. We have evaluated on heavily skewed workloads, where the buffer manager could be much more effective than evicting at random.

% Method heavily relies on the Delta Tree being small enough to be cached in memory or enough memory being available:
% Caching the Delta Tree in memory effectively is critical to keep read amplification low.
% Caching the Delta Tree in memory is cirtical to reduce write amplificaiton.

% this method could be applied to the table data as well, maybe where more writes happen than reads.

% Only use DeltaTree when updates are spread out over many pages. If updates are concentrated on few pages, it is better to write them out directly.