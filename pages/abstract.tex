\chapter{\abstractname}

% Why should you read this thesis? Why do we choose B-trees?
B-trees are the most used data structure in modern database systems, due to their efficient access patterns and excellent lookup performance on large volumes of data.
% What is the problem in B-trees? Why are random writes important?
However, B-trees perform suboptimally under random writes, a pervasive pattern for secondary indexes.
Such workloads introduce write amplification in B-trees, a phenomenon where the amount of data written to storage is significantly larger than that of logically changed data.
% Why is solving write amplification important?
As a result, B-trees suffer increased latency, reduced throughput, and premature device wear with write-intensive workloads.

% Why did previous attempts to solve the problem fail?
An alternative is \ac{LSMT}, which trade off low read performance for high write performance.
However, this trade makes \ac{LSMT} unsuitable for generic database systems that require more balanced performance characteristics.
Other attempts to reduce write amplification in B-trees either reduce concurrency, impact read performance, or rely on hardware-specific features, limiting their effectiveness and applicability.

% What is the proposed approach?
This thesis introduces a lightweight buffering layer that minimizes the frequency and volume of write operations to external storage by reducing write amplification.
We hereby enable high performance under random writes, while sustaining all the benefits of traditional B-trees.

% What are the results?
We implement the proposed structure, evaluate its performance under different workloads, and compare it against state-of-the-art methods.
Compared to traditional B-trees, our method achieves up to 70\% fewer page writes while maintaining comparable read performance.

% What are the implications?
These results suggest that write-aware B-tree optimizations can extend the lifespan of storage devices and significantly improve the efficiency of write-intensive applications; contributing to the broader effort of designing storage-efficient data structures suited for modern hardware.