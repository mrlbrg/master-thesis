\chapter{\abstractname}

% Why should you read this thesis? Why do we choose B-trees?
B-trees are one of the most widely used data structures in modern database systems because of their efficient access patterns and excellent lookup performance on large volumes of data.
% What is the problem in B-trees? Why are random writes important?
However, B-trees perform suboptimally under random writes, a pervasive pattern for secondary indices.
Such workloads introduce write amplification in B-trees, a condition in which the amount of data written to storage is significantly larger than that of logically changed data.
% Why is solving write amplification important?
As a result, B-trees suffer increased latency, reduced throughput, and premature device wear with write-intensive workloads.

% Why did previous attempts to solve the problem fail?
An alternative is \ac{LSMT}, which trade lower read performance for higher write performance.
However, this trade-off makes \ac{LSMT} unsuitable for general-purpose database systems that require more balanced performance characteristics.
Other attempts to reduce write amplification in B-trees either reduce concurrency, degrade read performance, or rely on hardware-specific features, thereby limiting their effectiveness and applicability.

% What is the proposed approach?
This thesis introduces a lightweight buffering layer that minimizes the frequency and volume of write operations to external storage.
By doing so, we reduce write amplification and enable high performance under random writes, while preserving the benefits of traditional B-trees.

% What are the results?
We implement the proposed structure, evaluate its performance, and analyze its trade-offs.
Compared to traditional B-trees, our method achieves up to 70\% fewer page writes while maintaining comparable read performance.

% What are the implications?
These results suggest that write-aware optimizations for B-trees can significantly improve the efficiency of write-intensive applications, contributing to the broader effort to design storage-efficient data structures suited for modern hardware.