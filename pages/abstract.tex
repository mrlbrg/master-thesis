\chapter{\abstractname}

% Why should you read this thesis? Why do we choose B-Trees?
B-Trees are the most used data structure in modern database systems, due to their efficient access patterns and excellent lookup performance on large volumes of data.
% What is the problem in B-Trees? Why are random writes important?
However, B-Trees perform suboptimal under random writes, a particularly common pattern for secondary indexes.
Such workloads introduce write amplification in B-Trees, a phenomenon where the amount of data written to storage is significantly larger than the amount of data that logically changed.
% Why is solving write amplification important?
As a result, B-Trees suffer increased latency, reduced throughput, and premature device wear with write-intensive workloads.

% Why did previous attempts to solve the problem fail?
As an alternative, \ac{LSMT} were proposed, which trade off low read performance for high write performance.
However, this trade makes \ac{LSMT} unsuitable for generic database systems that require excellent read performance.
Other attempts to reduce write amplification in B-Trees either reduce concurrency, impact read performance or rely on hardware-specific features, limiting their effectiveness and applicability.

% What is the proposed approach?
This thesis introduces a lightweight buffering layer that minimizes the frequency and volume of write operations to external storage by reducing write amplification.
We hereby enable high performance under random writes, while sustaining all the benefits of traditional B-Trees.

% What are the results?
We implement the proposed structure, evaluate its performance under different workloads, and compare it against state-of-the-art methods.
Compared to \ac{LSMT}, our approach offers [...].
Compared to traditional B-Trees, our method achieves [...] while maintaining excellent read performance.

% What are the implications?
These results suggest that write-aware B-Tree optimizations can extend the lifespan of storage devices and significantly improve the efficiency of write-intensive applications; contributing to the broader effort of designing storage-efficient data structures suited for modern hardware.